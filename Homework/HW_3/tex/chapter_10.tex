\section{Chapter 10}
\subsection{10.1}
For Exercises 4, 6 and 8, determine whether the graph shown has
directed or undirected edges, whether it has multiple edges,
and whether it has one or more loops. Use your answers to
determine the type of graph in Table 1 this graph is.
\begin{itemize}
      \item[4.] \text{} \\ %To get the top of graphic to align w/ top
            \includegraphics[scale = 0.7]{img/10_1_4_graph.png} \\
            \answer \\
            Undirected with multiple edges. So it's a multigraph.

      \item[6.] \text{} \\
            \includegraphics[scale = 0.7]{img/10_1_6_graph.png} a\\
            \answer\\
            Undirected with multiple edges. It's a multigraph.

      \item[8.] \text{} \\
            \includegraphics[scale = 0.7]{img/10_1_8_graph.png} \\
            \answer \\
            Directed with multiple edges and loops. It's a directed multigraph.

      \item[18.] Who can influence Fred and whom can Fred influence in
            the influence graph in Example 2? \\
            \includegraphics[scale = 0.7]{img/10_1_18_graph.png} \\
            \answer \\
            Fred can influence Brian, and he can be influenced by Yvonne and Deborah.
      \item[32.]  Which statements must be executed before $S_6$ is executed
            in the program in Example 8? (Use the precedence graph in Figure 10.)\\
            \includegraphics[scale = 0.7]{img/10_1_32_graph.png} \\
            \answer \\
            $S_1,\ S_2,\ S_3,\ S_4$
\end{itemize}

\subsection{10.2}
\begin{itemize}
      \item[6.] Show that the sum, over the set of people at a party, of
            the number of people a person has shaken hands with, is
            even. Assume that no one shakes his or her own hand. \\
            \answer \\
            The sum must be even becuase the total goes up by 2 each time someone shakes
            another person's hand because it goes up by one for each person. Therefore
            the final total must be even.

      \item[14.] What does the degree of a vertex in the Hollywood graph
            represent? What does the neighborhood of a vertex represent?
            What do the isolated and pendant vertices represent? \\
            \answer \\
            The degree represents the amount of actors that person has worked with.
            The neighborhood represents the set of specific actors they have worked
            with. The isolated vertices mean that the actor haven't worked with
            anyone. The pendant vertices represent the actors/actressses who have
            only done a movie with one other actor/actress.

      \item[20.]  Draw these graphs.
            \begin{enumerate}[a.]
                  \item $K_7$
                  \item $K_{1,8}$
                  \item $K_{4,4}$
                  \item $C_7$
                  \item $W_7$
                  \item $Q_4$
            \end{enumerate}
            \answer
            \begin{enumerate}[a.]
                  \item \text{}\\
                        \includegraphics[scale=0.7]{img/10_2_20a_graph.png}
                  \item \text{}\\
                        \includegraphics[scale=0.7]{img/10_2_20b_graph.png}
                  \item \text{}\\
                        \includegraphics[scale=0.7]{img/10_2_20c_graph.png}
                  \item \text{}\\
                        \includegraphics[scale=0.7]{img/10_2_20d_graph.png}
                  \item \text{}\\
                        \includegraphics[scale=0.7]{img/10_2_20e_graph.png}
                  \item \text{}\\ %TODO: Make 4D hypercube
                        \includegraphics[scale=0.7]{img/10_2_20e_graph.png}
            \end{enumerate}
      \item[58.] Find the union of the given pair of simple
            graphs. (Assume edges with the same endpoints are the same.) \\
            \includegraphics[]{img/10_2_58Q_graph.png}\\
            \answer \\
            \includegraphics[scale = 0.65]{img/10_2_58A_graph.png}



\end{itemize}

\subsection{10.3}
For 2 and 4 use an adjacency list to represent the given graph.
\begin{itemize}
      \item[2.] \text{}\\
            \includegraphics[scale = 0.7]{img/10_3_2_graph.png} \\
            \answer \vspace{1mm} \\
            \begin{tabular}{c|l}
                  Vertex     & Adjacent Vertices \\
                  \hline
                  \textit{a} & \textit{b, d}     \\
                  \textit{b} & \textit{a, d, e}  \\
                  \textit{c} & \textit{d, e}     \\
                  \textit{d} & \textit{a, b, c}  \\
                  \textit{e} & \textit{b, c}
            \end{tabular}

      \item[4.] \text{}\\
            \includegraphics[scale = 0.7]{img/10_3_4_graph.png} \\
            \answer \vspace{1mm}\\
            \begin{tabular}{c|l}
                  Vertex     & Adjacent Vertices   \\
                  \hline
                  \textit{a} & \textit{b, d}       \\
                  \textit{b} & \textit{a, c, d, e} \\
                  \textit{c} & \textit{b, c}       \\
                  \textit{d} & \textit{a, e}       \\
                  \textit{e} & \textit{c, e}
            \end{tabular}

      \item[6.]  Represent the graph in Exercise 2 with an adjacency matrix. \\
            \answer \vspace{1mm} \\
            $
                  \begin{bmatrix}
                        0 & 1 & 0 & 1 & 0 \\
                        1 & 0 & 0 & 1 & 1 \\
                        0 & 0 & 0 & 1 & 1 \\
                        1 & 1 & 1 & 0 & 0 \\
                        0 & 1 & 1 & 0 & 0
                  \end{bmatrix}
            $

      \item[8.] Represent the graph in Exercise 4 with an adjacency matrix. \\
            \answer \vspace{1mm} \\
            $
                  \begin{bmatrix}
                        0 & 1 & 0 & 1 & 0 \\
                        1 & 0 & 1 & 1 & 1 \\
                        0 & 1 & 1 & 0 & 0 \\
                        1 & 0 & 0 & 0 & 1 \\
                        0 & 0 & 1 & 0 & 1
                  \end{bmatrix}
            $

      \item[12.] Draw a graph with the given adjacency matrix. \\
            \answer \\
            \includegraphics[scale = 0.7]{img/10_3_12_graph.png}

      \item[14.] Represent given graph with an adjacency matrix. \\
            \includegraphics[scale = 0.7]{img/10_3_14_graph.png}\\
            \answer \vspace{1.5mm} \\
            $
                  \begin{bmatrix}
                        0 & 3 & 0 & 1 \\
                        3 & 0 & 1 & 0 \\
                        0 & 1 & 0 & 3 \\
                        1 & 0 & 3 & 0
                  \end{bmatrix}
            $

      \item[26.]  Use an incidence matrix to represent the graphs in Exercises 1 and 2.
            %TODO: Should probably label the edges.
            \begin{tasks}(2)
                  \task \text{}\\
                  \includegraphics[scale = 0.7]{img/10_3_1_graph.png}
                  \task \text{}\\
                  \includegraphics[scale = 0.7]{img/10_3_2_graph.png}
            \end{tasks}
            \answer \\
            \begin{enumerate}[a.]
                  \item $
                              \begin{bmatrix}
                                    1 & 1 & 0 & 0 & 1 \\
                                    0 & 1 & 1 & 0 & 0 \\
                                    1 & 0 & 0 & 1 & 0 \\
                                    0 & 0 & 1 & 1 & 1
                              \end{bmatrix}
                        $
                  \item $
                              \begin{bmatrix}
                                    1 & 1 & 0 & 0 & 0 & 0 \\
                                    0 & 1 & 1 & 0 & 1 & 0 \\
                                    0 & 0 & 0 & 1 & 0 & 1 \\
                                    1 & 0 & 1 & 1 & 0 & 0 \\
                                    0 & 0 & 0 & 0 & 1 & 1
                              \end{bmatrix}
                        $
            \end{enumerate}
\end{itemize}
In Exercises 38 and 42 determine whether the given pair of graphs
is isomorphic. Exhibit an isomorphism or provide a rigorous
argument that none exists.
\begin{itemize}
      \item[38.]\text{}\\
            \includegraphics[]{img/10_3_38_graph.png} \\
            \answer \\
            These graphs are isomorphic with a function $f$ that has $f(u_1) = v_1,\ f(u_2) =v_5,\
                  f(u_3) = v_2,\ f(u_4) = v_3,\ f(u_5) = v_4$
      \item[42.]\text{}\\
            \includegraphics[scale = 0.7]{img/10_3_42_graph.png} \\
            \answer \\
            They are not isomorphic because $deg(u_4) =3$ so it must map to $v_3$ since it is the only one in the second graph with
            the same degree. However it can't map to $v_3$ since it must also be adjacent to
            exactly one vertex of degree 4 but $v_3$ is adjacent to two degree 4 vertices.

\end{itemize}

\subsection{10.4}
%TODO: 10.4 – 2, 8, 14
\begin{itemize}
      \item[2.]  Does each of these lists of vertices form a path in the
            following graph? Which paths are simple? Which are circuits? What are
            the lengths of those that are paths?
            \begin{tasks}[item-format=\textit](2)
                  \task a, b, e, c, b
                  \task a, d, a, d, a
                  \task a, d, b, e, a
                  \task a, b, e, c, b, d, a
            \end{tasks}
            \includegraphics[scale = 0.7]{img/10_4_2_graph.png}\\
            \answer
            \begin{enumerate}[a.]
                  \item Simple path, length: 4
                  \item Not simple path, length: 4
                  \item Not a path
                  \item Not a path
            \end{enumerate}
      
      \item[8.]  What do the connected components of a collaboration
graph represent? \\
\answer \\
They represent the largest set of people who have worked together or there is 
a way to get from one person to another via the other people in the set.

\item[14.]  Find the strongly connected components of each of these
graphs.
\begin{enumerate}[a.]
      \item \text{} \\ 
      \includegraphics[scale=0.7]{img/10_4_14a_graph.png}

      \item \text{} \\ 
      \includegraphics[scale=0.7]{img/10_4_14b_graph.png}
      
      \item \text{} \\ 
      \includegraphics[scale=0.7]{img/10_4_14c_graph.png}
\end{enumerate}
\answer
\begin{enumerate}[a.]
      \item $\{a, b, e\},\ \{c\},\ \{d\}$
      \item $\{c,d,e\},\ \{a\},\ \{b\},\ \{f\}$
      \item $\{a, b, c, d, f, g, h, i\},\ \{e\}$
\end{enumerate}


\end{itemize}

\subsection{10.5}
\begin{itemize}
      \item[2.] Determine whether the given graph has an
Euler circuit. Construct such a circuit when one exists. If
no Euler circuit exists, determine whether the graph has an
Euler path and construct such a path if one exists. \\
\includegraphics[]{img/10_5_2_graph.png} \\
\answer \\
An Euler circuit exists since all vertices have an even degree. (\textit{a, b, e, d, g, h, f, i, h, e, f, c, b, d, a})
\end{itemize}