\section{Chapter 3}
\subsection{3.1}
\begin{itemize}
    \item[4.] Describe an algorithm that takes as input a list of n integers and produces as output the largest difference obtained by subtracting an integer in the list from the one following it.\\
          \answer \\
          Go through the list starting at first element ending at the second to last element and sent the max\_difference to zero. Take the absolute value of the difference between the element you are on and the next one. Compare the value to max\_difference and replace if it is larger.

    \item[12.] Describe an algorithm that uses only assignment statements that replaces the triple (x, y, z) with (y, z, x). What is the minimum number of assignment statements needed?\\
          \answer \\
          temp = z \\
          z = x \\
          x = y\\
          y = temp \vspace{3mm}\\
          Min is 4  statements

    \item[34.] Use the bubble sort to sort 6, 2, 3, 1, 5, 4, showing the lists obtained at each step.\\
          \answer \\
          2, 3, 1, 5, 6 \\
          2, 1, 3, 5, 6\\
          1, 2, 3, 5, 6\\
\end{itemize}

\subsection{3.2}
\begin{itemize}
    \item[2.] Determine whether each of these functions is $\mathcal{O}(x^2)$.
          \begin{enumerate}[a.]
              \item $f(x) = 17x + 11$
              \item $f(x) = x^2 + 1000$
              \item $f(x) = x \log{}x$
              \item $f(x) = x^4 / 2$
              \item $f(x) = 2^x$
              \item $f(x) = \lfloor x \rfloor \cdot \lceil x \rceil$
          \end{enumerate}
          \answer
          \begin{enumerate}[a.]
              \item Yes
              \item Yes
              \item Yes
              \item No
              \item No
              \item Yes
          \end{enumerate}
    \item[24.] Suppose that you have two different algorithms for solving a problem. To solve a problem of size $n$, the first algorithm uses exactly $n^2 \cdot 2^n$ operations and the second algorithm uses exactly $n!$ operations. As $n$ grows, which algorithm uses fewer operations?

          \answer \quad $n^2 \cdot 2^n$ grows faster as $n$ grows.
\end{itemize}

\subsection{3.3}
\begin{itemize}
    \item[2.] Give a big-$\mathcal{O}$ estimate for the number additions used in this segment of an algorithm.

          \begin{quote}
              $t := 0$\\
              $for i := 1 \text{ to } n$
              \begin{quote}
                  $\text{for $j$}:= 1 \text{ to } n$
                  \begin{quote}
                      $t := t + i + j$
                  \end{quote}
              \end{quote}
          \end{quote}

          \answer \quad $\mathcal{O}(n^2)$

    \item[16.] What is the largest $n$ for which one can solve within a day using an algorithm that requires $f(n)$ bit operations, where each bit operation is carried out in $10^{-11}$ seconds, with these functions $f(n)$?
          \begin{enumerate}[a.]
              \item $\log{} n$
              \item $1000n$
              \item $n^2$
              \item $1000n^2$
              \item $n^3$
              \item $2^n$
              \item $2^{2n}$
              \item $2^{2^n}$
          \end{enumerate}
          \answer \\
          Note: Max bit operations in a day is $8.64 \cdot 10^{15}$
          \begin{enumerate}[a.]
              \item $2^{(8.64*10^{15})}$
              \item $8.64*10^{12}$
              \item $92951600$
              \item $2939387$
              \item $205197$
              \item $52$
              \item $26$
              \item $5$
          \end{enumerate}
    \item[18.] How much time does an algorithm take to solve a problem of size $n$ if this algorithm uses $2n^2 + 2^n$ operations, each requiring $10^{-9}$ seconds, with these values of $n$?
    \begin{enumerate}[a.]
          \item 10
          \item 20
          \item 50
          \item 100
    \end{enumerate}
\answer
\begin{enumerate}[a.]
          \item $1.224 \cdot 10^{-6}$ s
          \item $0.001049376$ s
          \item $1125899.906847624$ s
          \item $1.2676506002282295 \cdot 10^{21}$ s
\end{enumerate}

\end{itemize}