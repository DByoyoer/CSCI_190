\section{Chapter 1}
\subsection{1.1}
\begin{itemize}
\item[2.]Which of these are propositions? What are the truth values of those that are propositions?
\begin {enumerate}[a.]
\item Do not pass go.
\item What time is it?
\item There are no black flies in Maine.
\item 4 + x = 5.
\item The moon is made of green cheese.
\item 2n $\geq$ 100.
\end{enumerate}
\textbf{Answer:} c and e are both propositions, and both of their truth values are false.
\item[4.]What is the negation of each of these propositions?
\begin{enumerate}[a.]
    \item Jennifer and Teja are friends.
    \item There are 13 items in a baker’s dozen.
    \item Abby sent more than 100 text messages every day.
    \item 121 is a perfect square.
\end{enumerate}
\textbf{Answer:}
\begin{enumerate}[a.]
    \item Jennifer and Teja are not friends.
    \item There aren’t 13 items in a baker’s dozen.
    \item Abby sent less than or equal to 100 text messages every day.
    \item 121 is not a perfect square.
\end{enumerate}
\item[6.]Suppose that Smartphone A has 256 MB RAM and 32 GB ROM, and the resolution of its camera is 8 MP; Smartphone B has 288 MB RAM and 64 GB ROM, and the resolution of its camera is 4 MP; and Smartphone C has 128 MB RAM and 32 GB ROM, and the resolution of
its camera is 5 MP. Determine the truth value of each of these propositions.
\begin{enumerate}[a.]
    \item Smartphone B has the most RAM of these three smartphones.
    \item Smartphone C has more ROM or a higher resolution camera than Smartphone B.
    \item Smartphone B has more RAM, more ROM, and a higher resolution camera than Smartphone A.
    \item If Smartphone B has more RAM and more ROM than Smartphone C, then it also has a higher resolution camera.
    \item Smartphone A has more RAM than Smartphone B if and only if Smartphone B has more RAM than Smartphone A.
\end{enumerate}
\textbf{Answer:}
\begin{enumerate}[a.]
    \item True
    \item True
    \item False
    \item False
    \item False
\end{enumerate}
\end{itemize}
\subsection{1.2}
\textbf{For exercises 2 \& 4, translate into propositional logic.}
\begin{itemize}
    \item [2.]You can see the movie only if you are over 18 years old or you have the permission of a parent. Express your answer in terms of m: “You can see the movie,” e: “You are over 18 years old,” and p: “You have the permission of a parent.”

          \textbf{Answer:} $m \to (e \lor p)$
    \item[4.]To use the wireless network in the airport you must pay the daily fee unless you are a subscriber to the service. Express your answer in terms of w: “You can use the wire-
          less network in the airport,” d: “You pay the daily fee,” and s: “You are a subscriber to the service.”

          \textbf{Answer:} $w \to (d \lor s)$
\end{itemize}

\subsection{1.3}
\begin{itemize}
    \item[6.]Use a truth table to verify the first De Morgan law

          \[
              \neg (p \land q) \equiv \neg p \lor \neg q
          \]
          \textbf{Answer:}
          \begin{longtable}[c]{|l|l|l|l|l|l|}
              \hline
              \textit{p} & \textit{q} & $\neg p$ & $\neg q$ & $\neg (p \land q)$ & $\neg p \lor \neg q$ \\
              \hline
              \endfirsthead
              T          & T          & F        & F        & F                  & F                    \\
              \hline
              T          & F          & F        & T        & T                  & T                    \\
              \hline
              F          & T          & T        & F        & T                  & T                    \\
              \hline
              F          & F          & T        & T        & T                  & T                    \\
              \hline
          \end{longtable}
    \item[8.]Use De Morgan’s laws to find the negation of each of the following statements
          \begin{enumerate}[a.]
              \item Kwame will take a job in industry or go to graduate school.
              \item Yoshiko knows Java and calculus.
              \item James is young and strong.
              \item Rita will move to Oregon or Washington.
          \end{enumerate}
          \textbf{Answer:}
          \begin{enumerate}[a.]
              \item Kwame will not take a job in industry and not go to graduate school.
              \item Yoshiko doesn’t know Java or doesn’t know calculus.
              \item James is not young or not strong.
              \item Rita will not move to Oregon and will not move to Washington.
          \end{enumerate}
    \item[10.]Show that each of these conditional statements is a tautology by using truth tables.
          \begin{enumerate}[a.]
              \item $[\neg p \land (p \lor q)] \to q$
              \item $[(p \to q) \land (q \to r)] \to (p \to r)$
              \item $[p \land (p \to q)] \to q$
              \item $[(p \lor q) \land (p \to r) \land (q \to r)] \to r$
          \end{enumerate}
          \textbf{Answer:}
          \begin{enumerate}[a.]
              \item
                    \begin{tabular}{|l|l|l|l|l|l|}
                        \hline
                        \textit{p} & \textit{q} & $\neg p$ & $p \lor q$ & $[\neg p \land (p \lor q)]$ & $[\neg p \land (p \lor q)] \to q$ \\
                        \hline
                        T          & T          & F        & T          & F                           & T                                 \\
                        \hline
                        T          & F          & F        & T          & F                           & T                                 \\
                        \hline
                        F          & T          & T        & T          & T                           & T                                 \\
                        \hline
                        F          & F          & T        & T          & T                           & T                                 \\
                        \hline
                    \end{tabular}
              \item
                    \begin{tabular}{|l|l|l|l|l|l|l|p{3cm}|}
                        \hline
                        \textit{p} & \textit{q} & \textit{r} & $p \to q$ & $q \to r$ & $p \to r$ & $[(p \to q) \land (q \to r)]$ & $[(p \to q) \land (q \to r)] \newline \to (p \to r)$ \\
                        \hline
                        T          & T          & T          & T         & T         & T         & T                             & T                                                    \\
                        \hline
                        T          & T          & F          & T         & F         & F         & F                             & T                                                    \\
                        \hline
                        T          & F          & T          & F         & T         & T         & T                             & T                                                    \\
                        \hline
                        T          & F          & F          & F         & T         & T         & F                             & T                                                    \\
                        \hline
                        F          & T          & T          & T         & T         & T         & T                             & T                                                    \\
                        \hline
                        F          & T          & F          & T         & F         & T         & F                             & T                                                    \\
                        \hline
                        F          & F          & T          & T         & T         & T         & T                             & T                                                    \\
                        \hline
                        F          & F          & F          & T         & T         & T         & T                             & T                                                    \\
                        \hline
                    \end{tabular}
              \item
                    \begin{tabular}{*{5}{|l}|}
                        \hline
                        \textit{p} & \textit{q} & $p \to q$ & $p \land (p \to q)$ & $[p \land (p \to q)] \to q$ \\
                        \hline
                        T          & T          & T         & T                   & T                           \\
                        \hline
                        T          & F          & F         & F                   & T                           \\
                        \hline
                        F          & T          & T         & F                   & T                           \\
                        \hline
                        F          & F          & T         & F                   & T                           \\
                        \hline
                    \end{tabular}
                    \newpage
              \item
                    \begin{tabular}{*{6}{|l} *{2}{|p{3.2cm}}|}
                        \hline
                        \textit{p} & \textit{q} & \textit{r} & $p \lor q$ & $p \to r$ & $q \to r$ & $(p \lor q) \land (p \to r) \land (q \to r) $ & $[(p \lor q) \land (p \to r) \land (q \to r)] \to r$
                        \\
                        \hline
                        T          & T          & T          & T          & T         & T         & T                                             & T                                                    \\
                        \hline
                        T          & T          & F          & T          & F         & F         & F                                             & T                                                    \\
                        \hline
                        T          & F          & T          & F          & T         & T         & T                                             & T                                                    \\
                        \hline
                        T          & F          & F          & F          & T         & T         & F                                             & T                                                    \\
                        \hline
                        F          & T          & T          & T          & T         & T         & T                                             & T                                                    \\
                        \hline
                        F          & T          & F          & T          & F         & T         & F                                             & T                                                    \\
                        \hline
                        F          & F          & T          & T          & T         & T         & T                                             & T                                                    \\
                        \hline
                        F          & F          & F          & T          & T         & T         & T                                             & T                                                    \\
                        \hline
                    \end{tabular}

          \end{enumerate}
    \item[12]Show that each conditional statement in Exercise 10 is a tautology without using truth tables.
          \begin{enumerate}[a.]
              \item $[\neg p \land (p \lor q)] \to q$
              \item $[(p \to q) \land (q \to r)] \to (p \to r)$
              \item $[p \land (p \to q)] \to q$
              \item $[(p \lor q) \land (p \to r) \land (q \to r)] \to r$
          \end{enumerate}
          \textbf{Answer:}
          \begin{enumerate}[a.]
              \item \begin{align*}
                        [\neg p \land (p \lor q)] \to q
                         & \equiv \neg[\neg p \land (p \lor q)] \lor q                         &  & \text{(Equivalence from table)}              \\
                         & \equiv [\neg(\neg p) \lor \neg(p \lor q)] \lor q                    &  & \text{(De Morgan's Law)}                     \\
                         & \equiv [p \lor (\neg p \land \neg q)] \lor q                        &  & \text{(Double negation and De Morgan's Law)} \\
                         & \equiv [(p \lor \neg p) \land (p \lor \neg q)] \lor q               &  & \text{(Distributive Law)}                    \\
                         & \equiv [\mathbf{T} \land (p \lor \neg q)] \lor q                    &  & \text{(Negation Law)}                        \\
                         & \equiv [(\mathbf{T} \land p) \lor (\mathbf{T} \land \neg q)] \lor q &  & \text{(Distributive Law)}                    \\
                         & \equiv (p \lor \neg q) \lor q                                       &  & \text{(Identity Law)}                        \\
                         & \equiv p \lor (\neg q \lor q)                                       &  & \text{(Associativity)}                       \\
                         & \equiv p \lor \mathbf{T}                                            &  & \text{(Negation Law)}                        \\
                         & \equiv \mathbf{T}                                                   &  & \text{(Domination Law)}                      \\
                    \end{align*}
              \item  \begin{align*}
                        [(p \to  q)  \land  (q \to r)] \to (p \to r)
                         & \equiv \neg [(p \to q) \land (q \to r)] \lor (p \to r)                                          &  & \text{(Equivalence from table)} \\
                         & \equiv [\neg (p \to q) \lor \neg (q \to r)] \lor (p \to r)                                      &  & \text{(De Morgan's Law)}        \\
                         & \equiv [(p \land \neg q) \lor (q \land \neg r)] \lor (p \to r)                                  &  & \text{(Equivalence from table)} \\
                         & \equiv [(p \land \neg q) \lor (q \land \neg r)] \lor (\neg p \lor r)                            &  & \text{(Equivalence from table)} \\
                         & \equiv [(p \land \neg q) \lor (\neg p \lor r)] \lor (q \land \neg r)                            &  & \text{(Commutative \&}          \\
                         &                                                                                                 &  & \text{Associative Law)}         \\
                         & \equiv [[(\neg p \lor r) \lor p] \land [(\neg p \lor r) \lor \neg q]] \lor (q \land \neg r)     &  & \text{(Distributive Law)}       \\
                         & \equiv [[(p \lor \neg p) \lor r] \land [(\neg p \lor r) \lor \neg q]] \lor (q \land \neg r)     &  & \text{(Commutative \&}          \\
                         &                                                                                                 &  & \text{Associative Law)}         \\
                         & \equiv [(\mathbf{T} \lor r) \land [(\neg p \lor r) \lor \neg q]] \lor (q \land \neg r)          &  & \text{(Negation Law)}           \\
                         & \equiv [\mathbf{T} \land [(\neg p \lor r) \lor \neg q]] \lor (q \land \neg r)                   &  & \text{(Domination Law)}         \\
                         & \equiv [(\neg p \lor r) \lor \neg q] \lor (q \land \neg r)                                      &  & \text{(Domination Law)}         \\
                         & \equiv [[(\neg p \lor r) \lor \neg q] \lor q] \land [[(\neg p \lor r) \lor \neg q] \lor \neg r] &  & \text{(Distributive Law)}       \\
                         & \equiv [(\neg p \lor r) \lor (\neg q \lor q)] \land [(\neg p \lor \neg q) \lor (r \lor \neg r)] &  & \text{(Commutative \&}          \\
                         &                                                                                                 &  & \text{Associative Law)}         \\
                         & \equiv [(\neg p \lor r) \lor \mathbf{T}] \land [(\neg p \lor \neg q) \lor \mathbf{T}]           &  & \text{(Negation Law)}           \\
                         & \equiv \mathbf{T} \land \mathbf{T}                                                              &  & \text{Domination Law}           \\
                         & \equiv \mathbf{T}                                                                               &  & \text{(Identity Law)}           \\
                    \end{align*}

              \item \begin{align*}
                        [p \land (p \to q)] \to q
                         & \equiv \neg [p \land (p \to q)] \lor q                     &  & \text{(Equivalence from table)} \\
                         & \equiv [\neg p \lor \neg (p \to q)] \lor q                 &  & \text{(De Morgan’s Law)}        \\
                         & \equiv [\neg p \lor (p \land \neg q)] \lor q               &  & \text{(Equivalence from table)} \\
                         & \equiv [(\neg p \lor p) \land (\neg p \lor \neg q)] \lor q &  & \text{(Distributive Law)}       \\
                         & \equiv [\mathbf{T} \land (\neg p \lor \neg q)] \lor q      &  & \text{(Negation Law)}           \\
                         & \equiv (\neg p \lor \neg q) \lor q                         &  & \text{(Identity Law)}           \\
                         & \equiv \neg p \lor (\neg q \lor q)                         &  & \text{(Associative Law)}        \\
                         & \equiv \neg p \lor \mathbf{T}                              &  & \text{(Negation Law)}           \\
                         & \equiv \mathbf{T}                                          &  & \text{(Domination Law)}
                    \end{align*}
              \item \begin{align*}
                        [(p \lor q) \land (p \to r) \land (q \to r)] \to r
                         & \equiv [(p \lor q) \land [(p \to r) \land (q \to r)]] \to r                       &  & \text{(Associative Law)}               \\
                         & \equiv [(p \lor q) \land [(p \lor q) \to r]] \to r                                &  & \text{(Equivalence from table)}        \\
                         & \equiv \neg [(p \lor q) \land [\neg (p \lor q) \lor r]] \lor r                    &  & \text{(Equivalence from table)}        \\
                         & \equiv [\neg (p \lor q) \lor \neg[\neg (p \lor q) \lor r]] \lor r                 &  & \text{(De Morgan’s Law)}               \\
                         & \equiv [\neg(p \lor q) \lor [(p \lor q) \land \neg r]] \lor r                     &  & \text{(De Morgan’s + Double Negation)} \\
                         & \equiv [\neg(p \lor q) \lor (p \lor q) \land [\neg(p \lor q) \lor \neg r]] \lor r &  & \text{(Distributive Law)}              \\
                         & \equiv [\mathbf{T} \land [\neg(p \lor q) \lor \neg r]] \lor r                     &  & \text{(Negation Law)}                  \\
                         & \equiv [\neg(p \lor q) \lor \neg r] \lor r                                        &  & \text{(Identity Law)}                  \\
                         & \equiv \neg(p \lor q) \lor (\neg r \lor r)                                        &  & \text{(Associative Law)}               \\
                         & \equiv \neg(p \lor q) \lor \mathbf{T}                                             &  & \text{(Negation Law)}                  \\
                         & \equiv \mathbf{T}                                                                 &  & \text{(Domination Law)}
                    \end{align*}
          \end{enumerate}
\end{itemize}

\subsection{1.4}
\begin{itemize}
\item[6.] Let N (x) be the statement “x has visited North Dakota,” where
the domain consists of the students in your school. Express each of these
quantifications in English.
\begin{enumerate}[a.]
\item $\exists xN(x)$
\item $\forall xN(x)$
\item $\neg\exists xN(x)$
\item $\exists x \neg N(x)$
\item $\neg\forall xN(x)$
\item $\forall x \neg N(x)$
\end {enumerate}
\textbf{Answer:}
\begin{enumerate}[a.]
    \item There exists a student in my school who has visited North Dakota.
    \item Every student in my school has visted North Dakota.
    \item No student in my school has visted North Dakota.
    \item At least one student in my school has not visted North Dakota.
    \item At least one student in my school has not visted North Dakota.
    \item No student in my school has visted North Dakota.
\end{enumerate}
\item[8.] Translate these statements into English, where R(x) is “x is a rabbit” and H (x) is “x hops” and the domain consists of all animals.
\begin {enumerate}[a.]
\item $\forall x(R(x) \to H(x))$
\item $\forall x(R(x) \land H(x))$
\item $\exists x(R(x) \to H(x))$
\item $\exists x(R(x) \land H(x))$
\end{enumerate}
\textbf{Answer:}
\begin{enumerate}[a.]
    \item Every rabbit hops.
    \item All animals are rabbits who hop.
    \item There exists an animal that if they are a rabbit, then they hop.
    \item Some rabbits hop.
\end{enumerate}
\item[14.] Determine the truth value of each of these statements if the domain consists of all real numbers.
\begin{enumerate}[a.]
    \item $\exists x(x^3 = -1)$
    \item $\exists x(x^4 < x^2)$
    \item $\forall x((-x)^2 = x^2)$
    \item $\forall x(2x > x)$
\end{enumerate}
\begin{enumerate}[a.]
    \item True, x = -1
    \item True, x = 0.1
    \item True, squaring returns a positive result
    \item False, x = -1 returns false.
\end{enumerate}
\end{itemize}

\subsection{1.5}
\begin{itemize}
    \item[2.] Translate these statements into English, where the domain for each
          variable consists of all real numbers.
          \begin{enumerate}[a.]
              \item $\exists x \forall y (xy = y)$
              \item $\forall x \forall y ((((x \geq 0) \land y < 0)) \to (x - y > 0)$
              \item $\forall x \forall y \exists z (x = y + z)$
          \end{enumerate}
          \textbf{Answer:}
          \begin{enumerate}[a.]
              \item There exists an $x$ such that for all $y$, $xy = y$, $x$ and $y$ are real.
              \item For all pairs of $x$ and $y$, both real, if $(x \geq 0)$ and $(y < 0)$, then $x - y > 0$.
              \item Let $x$, $y$ and $z$ be real numbers; for all pairs $x$ and $y$ there exists a $z$ such that $x = y + z$.
          \end{enumerate}

    \item[6.] Let C(x, y) mean that student x is enrolled in class y,
          where the domain for x consists of all students in your school and
          the domain for y consists of all classes being given at your school.
          Express each of these statements by a simple English sentence.
          \begin{enumerate}[a.]
              \item $C(\text{Randy Goldberg}, \text{CS } 252)$
              \item $\exists xC(x, \text{Math } 695)$
              \item $\exists yC(\text{Carol Sitea}, y)$
              \item $\exists x(C(x, \text{Math } 222) \land C(x, \text{CS } 252))$
              \item $\exists x \exists y \forall z ((x = y) \land (C(x, z) \to C(y, z)))$
              \item $\exists x \exists y \forall z ((x = y) \land (C(x, z) \leftrightarrow C(y, z)))$
          \end{enumerate}
          \textbf{Answer:}
          \begin{enumerate}[a.]
              \item Randy Goldberg is enrolled in CS 252.
              \item There exists a student in my school that is enrolled in Math 695.
              \item Carol Sitea is taking some class at my school.
              \item There exists a student who is enrolled in Math 22 and CS 252.
                    %TODO: Answer e and f
          \end{enumerate}
    \item[10.] Let $F(x, y)$ be the statement “x can fool y,” where the domain consists of
          all people in the world. Use quantifiers to express each of these statements.
          \begin{enumerate}[a.]
              \item Everybody can fool Fred.
              \item Evelyn can fool everybody.
              \item Everybody can fool somebody.
              \item There is no one who can fool everybody.
              \item Everyone can be fooled by somebody.
              \item No one can fool both Fred and Jerry.
              \item Nancy can fool exactly two people.
              \item There is exactly one person whom everybody can fool.
              \item No one can fool himself or herself.
              \item There is someone who can fool exactly one person besides himself or herself.
          \end{enumerate}
          \textbf{Answer:}
          \begin{enumerate}[a.]
              \item $\forall xF(x, \text{Fred})$
              \item $\forall xF(\text{Evelyn}, x)$
              \item $\forall x \exists yF(x,y)$
              \item $\forall x \exists y \neg F(x,y)$
              \item $\forall x \exists y F(y, x)$
              \item $\forall x(\neg F(x, \text{Fred}) \lor \neg F(x, \text{Jerry}))$
              \item $\exists x \exists y (x \neq y \land F(\text{Nancy}, x) \land F(\text{Nancy}, y))$
              \item $\exists x \forall y (F(y, x) \land \forall zF(F(y,z) \to z = x))$
              \item $\forall x \neg F(x, x)$
              \item $\exists x \exists y (x \neq y \land F(x, y) \land \forall zF(x,z) \to z = y)$
          \end{enumerate}
\end{itemize}

\subsection{1.6}
\begin{itemize}
    \item[2.] Find the argument form for the following argument and determine whether it is valid. Can we conclude that the conclusion is true if the premises are true?
          \vspace{5mm} %Need the separation or else shit goes weird

          \begin{tabular}{rl}
                           & If George does not have eight legs, then he is not a spider. \\
                           & George is a spider                                           \\
              \hline
              $\therefore$ & George has eight legs \vspace{3mm}                           %Need to find out why it has to go here
          \end{tabular}

          \textbf{Answer: }
          Yes the conclusion is true because of modus tollens.

    \item[4.] What rule of inference is used in each of these arguments?
          \begin{enumerate}[a.]
              \item Kangaroos live in Australia and are marsupials. Therefore, kangaroos are marsupials.
              \item It is either hotter than 100 degrees today or the pollution is dangerous. It is less than 100 degrees outside today. Therefore, the pollution is dangerous.
              \item Linda is an excellent swimmer. If Linda is an excellent swimmer, then she can work as a lifeguard. Therefore, Linda can work as a lifeguard.
              \item Steve will work at a computer company this summer. Therefore, this summer Steve will work at a computer company or he will be a beach bum.
              \item If I work all night on this homework, then I can answer all the exercises. If I answer all the exercises, I will understand the material. Therefore, if I work all night on this homework, then I will understand the material.
          \end{enumerate}
          \textbf{Answer:}
          \begin{enumerate}[a.]
              \item Simplification
              \item Disjunctive syllogism
              \item Modus ponens
              \item Addition
              \item Hypothetical syllogism
          \end{enumerate}

    \item[8] What rules of inference are used in this argument? “No man is an island. Manhattan is an island. Therefore, Manhattan is not a man.”

          \textbf{Answer:} Modus ponens
\end{itemize}

\subsection{1.7}
\begin{itemize}
    \item[6.] Use a direct proof to show that the product of two odd numbers is odd.
          \begin{proof}
              Let $x$ and $y$ be odd. This means that $x = 2k + 1$, $y = 2s + 1$; k, s $\in\mathbb{Z}$. \\
              Then:
              \begin{align*}
                  x*y & = (2k + 1) \cdot (2s+1) \\
                      & = 4ks + 2k + 2s + 1     \\
                      & = 2(2ks + k + s) + 1    \\
                      & = 2r + 1
              \end{align*}
              Which is an odd number where $r = 2ks + k + s$.
          \end{proof}

    \item[18.] Prove that if n is an integer and 3n + 2 is even, then n is even using
          \begin{enumerate}[a.]
              \item a proof by contraposition.
              \item a proof by contradiction.
          \end{enumerate}
          \begin{enumerate}[a.]
              \item
                    \begin{proof}
                        Assume that n is odd, so n = 2k  + 1
                        Then:
                        \begin{align*}
                            3n + 2 & = 3(2k + 1)  + 2 \\
                                   & = 6k + 3 + 2     \\
                                   & = 6k + 4 + 1     \\
                                   & = 2(3k + 2) + 1  \\
                                   & = 2r + 1
                        \end{align*}
                        Where $r \in \mathbb{Z}$\\
                        So when n is odd then, 3n + 2 is odd is true, and the contrapositive that when 3n + 2 is even, then n is even is also true.
                    \end{proof}

              \item
                    \begin{proof}
                        Assume that $3n + 2$ is even and $n$ is odd. Because $n$ is odd there is an integer $k$ such that $n = 2k + 1$
                        Then:
                        \begin{align*}
                            3n + 2 & = 3(2k + 1)  + 2 \\
                                   & = 6k + 3 + 2     \\
                                   & = 6k + 4 + 1     \\
                                   & = 2(3k + 2) + 1  \\
                                   & = 2r + 1         \\
                        \end{align*}
                        So $3n + 2 = 2r + 1$ where $r = 3k + 1$. Which shows $3n + 2$ is odd.
                        We have 3n + 2 both even and odd which is a contradiction, which means our assumption that n is odd is false thus n is even.
                    \end{proof}
          \end{enumerate}
    
    \item[24.] Show that at least three of any 25 days chosen must fall in the same month of the year.
          \begin{proof}
            Assume that no more than 2 of any 25 days chosen must fall in the same month of the year.
             Then because we have 12 months in a year at most 24 days can be chosen as each month can
             have at most 2 days, however this contradicts the premise that there are 25 days chosen so at least 3 of any 25 days must fall in the same month of the year
          \end{proof}
    \item[28.] Prove that $m^2 = n^2$ if and only if $m = n$ or $m = -n$ 
          \begin{proof}
            ($\leftarrow$) WTS: $(m=n \lor m=-n) \to m^2=n^2$ \\
            Let $m = n$ then $m^2 = m \cdot m = n \cdot n = n^2$ \\
            Let $m = -n$, then $m^2 = m \cdot m = -n \cdot -n = n^2$\\
            ($\to$) WTS: $m^2 = n^2 \to (m = n \lor m = -n)$\\ 
            Assume $m \neq n \text{ and } m \neq -n$
            \begin{align*}
            m \neq n && \text{Premise}\\
            m \cdot m \neq n \cdot m  &&\text{Multiply both sides by m}\\
            m^2 \neq m \cdot n \neq n \cdot n \neq n^2 &&\text{True from premise}\\
            m \neq -n\\
            m \cdot m \neq m \neq -n\\
            m^2 \neq m \cdot -n \neq -n \cdot -n \neq n^2 &&\text{Same steps as above}
            \end{align*}
            Contrapositive is true so the original statement is also true.\\
            Have shown both implications is to so the biconditional is true.
          \end{proof}
\end{itemize}