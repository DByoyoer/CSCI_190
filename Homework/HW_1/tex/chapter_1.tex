\section{Chapter 1}
\subsection{1.1}
\begin{itemize}
\item[2.]Which of these are propositions? What are the truth values of those that are propositions?
\begin {enumerate}[a.]
\item Do not pass go.
\item What time is it?
\item There are no black flies in Maine.
\item 4 + x = 5.
\item The moon is made of green cheese.
\item 2n $\geq$ 100.
\end{enumerate}
\textbf{Answer:} c and e are both propositions, and both of their truth values are false. 
\item[4.]What is the negation of each of these propositions?
\begin{enumerate}[a.]
\item Jennifer and Teja are friends.
\item There are 13 items in a baker’s dozen.
\item Abby sent more than 100 text messages every day.
\item 121 is a perfect square.
\end{enumerate}
\textbf{Answer:}
\begin{enumerate}[a.]
\item Jennifer and Teja are not friends.
\item There aren’t 13 items in a baker’s dozen.
\item Abby sent less than or equal to 100 text messages every day.
\item 121 is not a perfect square.
\end{enumerate}
\item[6.]Suppose that Smartphone A has 256 MB RAM and 32 GB ROM, and the resolution of its camera is 8 MP; Smartphone B has 288 MB RAM and 64 GB ROM, and the resolution of its camera is 4 MP; and Smartphone C has 128 MB RAM and 32 GB ROM, and the resolution of
its camera is 5 MP. Determine the truth value of each of these propositions.
\begin{enumerate}[a.]
\item Smartphone B has the most RAM of these three smartphones.
\item Smartphone C has more ROM or a higher resolution camera than Smartphone B.
\item Smartphone B has more RAM, more ROM, and a higher resolution camera than Smartphone A.
\item If Smartphone B has more RAM and more ROM than Smartphone C, then it also has a higher resolution camera.
\item Smartphone A has more RAM than Smartphone B if and only if Smartphone B has more RAM than Smartphone A.
\end{enumerate}
\textbf{Answer:}
\begin{enumerate}[a.]
    \item True
    \item True
    \item False
    \item False
    \item False
\end{enumerate}
\end{itemize}
\subsection{1.2}
\textbf{For exercises 2 \& 4, translate into propositional logic.}
\begin{itemize}
    \item [2.]You can see the movie only if you are over 18 years old or you have the permission of a parent. Express your answer in terms of m: “You can see the movie,” e: “You are over 18 years old,” and p: “You have the permission of a parent.”

    \textbf{Answer:} $m \rightarrow (e \lor p)$
    \item[4.]To use the wireless network in the airport you must pay the daily fee unless you are a subscriber to the service. Express your answer in terms of w: “You can use the wire-
less network in the airport,” d: “You pay the daily fee,” and s: “You are a subscriber to the service.”

    \textbf{Answer:} $w \rightarrow (d \lor s)$
\end{itemize}

\subsection{1.3}
\begin{itemize}
 \item[6.]Use a truth table to verify the first De Morgan law

\[
    \neg (p \land q) \equiv \neg p \lor \neg q 
\]
\textbf{Answer:}
\begin{longtable}[c]{|l|l|l|l|l|l|}
    \hline
   \textit{p} & \textit{q} & $\neg p$ & $\neg q$ & $\neg (p \land q)$ & $\neg p \lor \neg q$ \\
   \hline
   \endfirsthead
   T & T & F & F & F & F \\
   \hline
   T & F & F & T & T & T \\
   \hline
   F & T & T & F & T & T \\
   \hline
   F & F & T & T & T & T \\
   \hline
\end{longtable}
\item[8.]Use De Morgan’s laws to find the negation of each of the following statements 
\begin{enumerate}[a.]
\item Kwame will take a job in industry or go to graduate school.
\item Yoshiko knows Java and calculus.
\item James is young and strong.
\item Rita will move to Oregon or Washington.
\end{enumerate}   

\end{itemize}
