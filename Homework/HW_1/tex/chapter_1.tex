\section{Chapter 1}
\subsection{1.1}
\begin{itemize}
\item[2.]Which of these are propositions? What are the truth values of those that are propositions?
\begin {enumerate}[a.]
\item Do not pass go.
\item What time is it?
\item There are no black flies in Maine.
\item 4 + x = 5.
\item The moon is made of green cheese.
\item 2n $\geq$ 100.
\end{enumerate}
\textbf{Answer:} c and e are both propositions, and both of their truth values are false.
\item[4.]What is the negation of each of these propositions?
\begin{enumerate}[a.]
    \item Jennifer and Teja are friends.
    \item There are 13 items in a baker’s dozen.
    \item Abby sent more than 100 text messages every day.
    \item 121 is a perfect square.
\end{enumerate}
\textbf{Answer:}
\begin{enumerate}[a.]
    \item Jennifer and Teja are not friends.
    \item There aren’t 13 items in a baker’s dozen.
    \item Abby sent less than or equal to 100 text messages every day.
    \item 121 is not a perfect square.
\end{enumerate}
\item[6.]Suppose that Smartphone A has 256 MB RAM and 32 GB ROM, and the resolution of its camera is 8 MP; Smartphone B has 288 MB RAM and 64 GB ROM, and the resolution of its camera is 4 MP; and Smartphone C has 128 MB RAM and 32 GB ROM, and the resolution of
its camera is 5 MP. Determine the truth value of each of these propositions.
\begin{enumerate}[a.]
    \item Smartphone B has the most RAM of these three smartphones.
    \item Smartphone C has more ROM or a higher resolution camera than Smartphone B.
    \item Smartphone B has more RAM, more ROM, and a higher resolution camera than Smartphone A.
    \item If Smartphone B has more RAM and more ROM than Smartphone C, then it also has a higher resolution camera.
    \item Smartphone A has more RAM than Smartphone B if and only if Smartphone B has more RAM than Smartphone A.
\end{enumerate}
\textbf{Answer:}
\begin{enumerate}[a.]
    \item True
    \item True
    \item False
    \item False
    \item False
\end{enumerate}
\end{itemize}
\subsection{1.2}
\textbf{For exercises 2 \& 4, translate into propositional logic.}
\begin{itemize}
    \item [2.]You can see the movie only if you are over 18 years old or you have the permission of a parent. Express your answer in terms of m: “You can see the movie,” e: “You are over 18 years old,” and p: “You have the permission of a parent.”

          \textbf{Answer:} $m \to (e \lor p)$
    \item[4.]To use the wireless network in the airport you must pay the daily fee unless you are a subscriber to the service. Express your answer in terms of w: “You can use the wire-
          less network in the airport,” d: “You pay the daily fee,” and s: “You are a subscriber to the service.”

          \textbf{Answer:} $w \to (d \lor s)$
\end{itemize}

\subsection{1.3}
\begin{itemize}
    \item[6.]Use a truth table to verify the first De Morgan law

          \[
              \neg (p \land q) \equiv \neg p \lor \neg q
          \]
          \textbf{Answer:}
          \begin{longtable}[c]{|l|l|l|l|l|l|}
              \hline
              \textit{p} & \textit{q} & $\neg p$ & $\neg q$ & $\neg (p \land q)$ & $\neg p \lor \neg q$ \\
              \hline
              \endfirsthead
              T          & T          & F        & F        & F                  & F                    \\
              \hline
              T          & F          & F        & T        & T                  & T                    \\
              \hline
              F          & T          & T        & F        & T                  & T                    \\
              \hline
              F          & F          & T        & T        & T                  & T                    \\
              \hline
          \end{longtable}
    \item[8.]Use De Morgan’s laws to find the negation of each of the following statements
          \begin{enumerate}[a.]
              \item Kwame will take a job in industry or go to graduate school.
              \item Yoshiko knows Java and calculus.
              \item James is young and strong.
              \item Rita will move to Oregon or Washington.
          \end{enumerate}
          \textbf{Answer:}
          \begin{enumerate}[a.]
              \item Kwame will not take a job in industry and not go to graduate school.
              \item Yoshiko doesn’t know Java or doesn’t know calculus.
              \item James is not young or not strong.
              \item Rita will not move to Oregon and will not move to Washington.
          \end{enumerate}
    \item[10.]Show that each of these conditional statements is a tautology by using truth tables.
          \begin{enumerate}[a.]
              \item $[\neg p \land (p \lor q)] \to q$
              \item $[(p \to q) \land (q \to r)] \to (p \to r)$
              \item $[p \land (p \to q)] \to q$
              \item $[(p \lor q) \land (p \to r) \land (q \to r)] \to r$
          \end{enumerate}
          \textbf{Answer:}
          \begin{enumerate}[a.]
              \item
                    \begin{tabular}{|l|l|l|l|l|l|}
                        \hline
                        \textit{p} & \textit{q} & $\neg p$ & $p \lor q$ & $[\neg p \land (p \lor q)]$ & $[\neg p \land (p \lor q)] \to q$ \\
                        \hline
                        T          & T          & F        & T          & F                           & T                                 \\
                        \hline
                        T          & F          & F        & T          & F                           & T                                 \\
                        \hline
                        F          & T          & T        & T          & T                           & T                                 \\
                        \hline
                        F          & F          & T        & T          & T                           & T                                 \\
                        \hline
                    \end{tabular}
                    \\
                    \\
              \item
                    \begin{tabular}{|l|l|l|l|l|l|l|p{3cm}|}
                        \hline
                        \textit{p} & \textit{q} & \textit{r} & $p \to q$ & $q \to r$ & $p \to r$ & $[(p \to q) \land (q \to r)]$ & $[(p \to q) \land (q \to r)] \newline \to (p \to r)$ \\
                        \hline
                        T          & T          & T          & T         & T         & T         & T                             & T                                                    \\
                        \hline
                        T          & T          & F          & T         & F         & F         & F                             & T                                                    \\
                        \hline
                        T          & F          & T          & F         & T         & T         & T                             & T                                                    \\
                        \hline
                        T          & F          & F          & F         & T         & T         & F                             & T                                                    \\
                        \hline
                        F          & T          & T          & T         & T         & T         & T                             & T                                                    \\
                        \hline
                        F          & T          & F          & T         & F         & T         & F                             & T                                                    \\
                        \hline
                        F          & F          & T          & T         & T         & T         & T                             & T                                                    \\
                        \hline
                        F          & F          & F          & T         & T         & T         & T                             & T                                                    \\
                        \hline
                    \end{tabular}
                    \\
                    \\
              \item
                    \begin{tabular}{*{5}{|l}|}
                        \hline
                        \textit{p} & \textit{q} & $p \to q$ & $p \land (p \to q)$ & $[p \land (p \to q)] \to q$ \\
                        \hline
                        T          & T          & T         & T                   & T                           \\
                        \hline
                        T          & F          & F         & F                   & T                           \\
                        \hline
                        F          & T          & T         & F                   & T                           \\
                        \hline
                        F          & F          & T         & F                   & T                           \\
                        \hline
                    \end{tabular}
                    \newpage
              \item
                    \begin{tabular}{*{6}{|l} *{2}{|p{3.2cm}}|}
                        \hline
                        \textit{p} & \textit{q} & \textit{r} & $p \lor q$ & $p \to r$ & $q \to r$ & $(p \lor q) \land (p \to r) \land (q \to r) $ & $[(p \lor q) \land (p \to r) \land (q \to r)] \to r$
                        \\
                        \hline
                        T          & T          & T          & T          & T         & T         & T                                              & T                                                    \\
                        \hline
                        T          & T          & F          & T          & F         & F         & F                                              & T                                                    \\
                        \hline
                        T          & F          & T          & F          & T         & T         & T                                              & T                                                    \\
                        \hline
                        T          & F          & F          & F          & T         & T         & F                                              & T                                                    \\
                        \hline
                        F          & T          & T          & T          & T         & T         & T                                              & T                                                    \\
                        \hline
                        F          & T          & F          & T          & F         & T         & F                                              & T                                                    \\
                        \hline
                        F          & F          & T          & T          & T         & T         & T                                              & T                                                    \\
                        \hline
                        F          & F          & F          & T          & T         & T         & T                                              & T                                                    \\
                        \hline
                    \end{tabular}

          \end{enumerate}
    \item[12]Show that each conditional statement in Exercise 10 is a tautology without using truth tables.
          \begin{enumerate}[a.]
              \item $[\neg p \land (p \lor q)] \to q$
              \item $[(p \to q) \land (q \to r)] \to (p \to r)$
              \item $[p \land (p \to q)] \to q$
              \item $[(p \lor q) \land (p \to r) \land (q \to r)] \to r$
          \end{enumerate}
          \textbf{Answer:}
          \begin{enumerate}[a.]
              \item \begin{align*}
                        [\neg p \land (p \lor q)] \to q
                         & \equiv \neg[\neg p \land (p \lor q)] \lor q                         &  & \text{(Equivalence from table)}              \\
                         & \equiv [\neg(\neg p) \lor \neg(p \lor q)] \lor q                    &  & \text{(De Morgan's Law)}                     \\
                         & \equiv [p \lor (\neg p \land \neg q)] \lor q                        &  & \text{(Double negation and De Morgan's Law)} \\
                         & \equiv [(p \lor \neg p) \land (p \lor \neg q)] \lor q               &  & \text{(Distributive Law)}                    \\
                         & \equiv [\mathbf{T} \land (p \lor \neg q)] \lor q                    &  & \text{(Negation Law)}                        \\
                         & \equiv [(\mathbf{T} \land p) \lor (\mathbf{T} \land \neg q)] \lor q &  & \text{(Distributive Law)}                    \\
                         & \equiv (p \lor \neg q) \lor q                                       &  & \text{(Identity Law)}                        \\
                         & \equiv p \lor (\neg q \lor q)                                       &  & \text{(Associativity)}                       \\
                         & \equiv p \lor \mathbf{T}                                            &  & \text{(Negation Law)}                        \\
                         & \equiv \mathbf{T}                                                   &  & \text{(Domination Law)}                      \\
                    \end{align*}
              \item  \begin{align*}
                        [(p \to  q)  \land  (q \to r)] \to (p \to r)
                         & \equiv \neg [(p \to q) \land (q \to r)] \lor (p \to r)                                          &  & \text{(Equivalence from table)}         \\
                         & \equiv [\neg (p \to q) \lor \neg (q \to r)] \lor (p \to r)                                      &  & \text{(De Morgan's Law)}                \\
                         & \equiv [(p \land \neg q) \lor (q \land \neg r)] \lor (p \to r)                                  &  & \text{(Equivalence from table)}         \\
                         & \equiv [(p \land \neg q) \lor (q \land \neg r)] \lor (\neg p \lor r)                            &  & \text{(Equivalence from table)}         \\
                         & \equiv [(p \land \neg q) \lor (\neg p \lor r)] \lor (q \land \neg r)                            &  & \text{(Commutative \& Associative Law)} \\
                         & \equiv [[(\neg p \lor r) \lor p] \land [(\neg p \lor r) \lor \neg q]] \lor (q \land \neg r)     &  & \text{(Distributive Law)}               \\
                         & \equiv [[(p \lor \neg p) \lor r] \land [(\neg p \lor r) \lor \neg q]] \lor (q \land \neg r)     &  & \text{(Commutative \& Associative Law)} \\
                         & \equiv [(\mathbf{T} \lor r) \land [(\neg p \lor r) \lor \neg q]] \lor (q \land \neg r)          &  & \text{(Negation Law)}                   \\
                         & \equiv [\mathbf{T} \land [(\neg p \lor r) \lor \neg q]] \lor (q \land \neg r)                   &  & \text{(Domination Law)}                 \\
                         & \equiv [(\neg p \lor r) \lor \neg q] \lor (q \land \neg r)                                      &  & \text{(Domination Law)}                 \\
                         & \equiv [[(\neg p \lor r) \lor \neg q] \lor q] \land [[(\neg p \lor r) \lor \neg q] \lor \neg r] &  & \text{(Distributive Law)}               \\
                         & \equiv [(\neg p \lor r) \lor (\neg q \lor q)] \land [(\neg p \lor \neg q) \lor (r \lor \neg r)] &  & \text{(Commutative \& Associative Law)} \\
                         & \equiv [(neg p \lor r) \lor \mathbf{T}] \land [(\neg p \lor \neg q) \lor \mathbf{T}]            &  & \text{(Negation Law)}                   \\
                         & \equiv \mathbf{T} \land \mathbf{T}                                                              &  & \text{Domination Law}                   \\
                         & \equiv \mathbf{T}                                                                               &  & \text{(Identity Law)}
                    \end{align*}
              \item \begin{align*}
                        [p \land (p \to q)] \to q
                         & \equiv \neg [p \land (p \to q)] \lor q                     &  & \text{(Equivalence from table)} \\
                         & \equiv [\neg p \lor \neg (p \to q)] \lor q                 &  & \text{(De Morgan’s Law)}        \\
                         & \equiv [\neg p \lor (p \land \neg q)] \lor q               &  & \text{(Equivalence from table)} \\
                         & \equiv [(\neg p \lor p) \land (\neg p \lor \neg q)] \lor q &  & \text{(Distributive Law)}       \\
                         & \equiv [\mathbf{T} \land (\neg p \lor \neg q)] \lor q      &  & \text{(Negation Law)}           \\
                         & \equiv (\neg p \lor \neg q) \lor q                         &  & \text{(Identity Law)}           \\
                         & \equiv \neg p \lor (\neg q \lor q)                         &  & \text{(Associative Law)}        \\
                         & \equiv \neg p \lor \mathbf{T}                              &  & \text{(Negation Law)}           \\
                         & \equiv \mathbf{T}                                          &  & \text{(Domination Law)}
                    \end{align*}
              \item \begin{align*}
                        [(p \lor q) \land (p \to r) \land (q \to r)] \to r
                        & \equiv [(p \lor q) \land [(p \to r) \land (q \to r)]] \to r && \text{(Associative Law)} \\
                        & \equiv [(p \lor q) \land [(p \lor q) \to r]] \to r && \text{(Equivalence from table)} \\ 
                        & \equiv \neg [(p \lor q) \land [\neg (p \lor q) \lor r]] \lor r && \text{(Equivalence from table)} \\
                        & \equiv [\neg (p \lor q) \lor \neg[\neg (p \lor q) \lor r]] \lor r && \text{(De Morgan’s Law)} \\
                        & \equiv [\neg(p \lor q) \lor [(p \lor q) \land \neg r]] \lor r && \text{(De Morgan’s + Double Negation)} \\
                        & \equiv [\neg(p \lor q) \lor (p \lor q) \land [\neg(p \lor q) \lor \neg r]] \lor r &&\text{(Distributive Law)} \\
                        & \equiv [\mathbf{T} \land [\neg(p \lor q) \lor \neg r]] \lor r && \text{(Negation Law)} \\
                        & \equiv [\neg(p \lor q) \lor \neg r] \lor r && \text{(Identity Law)} \\
                        & \equiv \neg(p \lor q) \lor (\neg r \lor r) &&\text{(Associative Law)} \\
                        & \equiv \neg(p \lor q) \lor \mathbf{T} && \text{(Negation Law)} \\
                        & \equiv \mathbf{T}  && \text{(Domination Law)}
                    \end{align*}
          \end{enumerate}

\end{itemize}
