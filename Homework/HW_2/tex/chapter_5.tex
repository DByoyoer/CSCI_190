\section{Chapter 5}
\subsection{5.1}
\begin{itemize}
    \item [4.] Let $P(n)$ be the statement that $1^3 + 2^3 +\ldots + n^3 = (n(n + 1)/2)^2$ for the positive integer $n$.
          \begin{enumerate}[a.]
              \item What is the statement P (1)?
              \item Show that P (1) is true, completing the basis step of the proof.
              \item What is the inductive hypothesis?
              \item What do you need to prove in the inductive step?
              \item Complete the inductive step, identifying where you use the inductive hypothesis.
              \item Explain why these steps show that this formula is true whenever n is a positive integer.
          \end{enumerate}
          \answer
          \begin{enumerate}[a.]
              \item $P(1):1^3 = 1(1(1+1)/2)^2$
              \item $(1(1+1)/2)^2 = (2/2)^2 = 1 = 1^3 = 1$
              \item Inductive Hypothesis: $ 1^3 + 2^3 + \ldots + k^3 = (k(k+1)/2)^2$
              \item Need to prove that $P(k) \to P(k+1)$ for $k \geq 1$
              \item
                    \begin{align*}
                        \sum_{n=1}^{k+1} i^3 & = \sum_{n=1}^{k} i^3 + (k+1)^3                                                    \\
                                             & = \left(\frac{k(k+1)}{2}\right)^2 + (k+1)^3           &  & (Inductive Hypothesis) \\
                                             & =\frac{k^4 + 2k^3 + k^2}{4}+ \frac{4(k+1)^3}{4}                                   \\
                                             & = \frac{k^4 + 2k^3 + k^2 + 4(k+1)^3}{4}                                           \\
                                             & = \frac{k^4 + 2k^3 + k^2 + 4(k^3 + 3k^2 + 3k + 1)}{4}                             \\
                                             & = \frac{k^3 + 2k^3 + k^2 + 4k^3 + 12k^2 + 12k + 4}{4}                             \\
                                             & = \frac{k^4 + 6k^3 + 13k^2 + 12k + 4}{4}                                          \\
                                             & = \frac{(k^2+3k+2)(k^2+3k+2)}{4}                                                  \\
                                             & = \frac{(k^2+3k+2)^2}{2^2}                                                        \\
                                             & = \frac{((k+1)(k+2))^2}{2^2}                                                      \\
                                             & = \left(\frac{(k+1)(k+2)}{2}\right)^2                                             \\
                                             & = \left(\frac{(k+1)((k+1)+1)}{2}\right)^2
                    \end{align*}
              \item Completed the base and inductive step, so by the principle of mathematical induction the statement is true for all positive integer $n$.
          \end{enumerate}

    \item[6.]  Prove that $1 \cdot 1! + 2 \cdot 2! + \ldots + n \cdot n! = (n + 1)! - 1$ whenever n is a positive integer. \\
          \answer
          \begin{proof}
              (Base Case) If $n=1$ then the left side is $1 \cdot 1! = 1$ and the left side is $(2)! - 1=1$ so the formula holds for n = 1.\\
              (Inductive Hypothesis) Assume that for $k \geq 1$ that the formula is true, that is $1 \cdot 1! + 2 \cdot 2! + \ldots + k \cdot k! = (k + 1)! - 1$. \\
              (Inductive Step) Let $n = k + 1$ Then:
              \begin{align*}
                  \sum_{n=1}^{k+1} i \cdot i! & = (k+1)(k+1)! + \sum_{n=1}^{k} i \cdot i!                                       \\
                                              & = (k+1)(k+1)! + (k+1)! -1                 &  & \text{(By inductive hypothesis)} \\
                                              & = (k+2)(k+1)! -1                                                                \\
                                              & = (k+2)! - 1
              \end{align*}
              This is the right side that we want so it holds for n = k +1. Thus by the principle of mathematical induction the theorem holds for all $n \in \mathbb{N}$.
          \end{proof}
\end{itemize}


\subsection{5.2}
\begin{itemize}
    \item[2.]  Use strong induction to show that all dominoes fall in an infinite arrangement of dominoes if you know that the
          first three dominoes fall, and that when a domino falls,
          the domino three farther down in the arrangement also
          falls. \\
          \answer
          \begin{proof}
              Since we know that the first dominoes will fall we will assume that the
              first $k$ dominoes will fall. If $k \leq 3$ then we already know that those
              will fall from what's given. We know that the $k - 2$ will fall by the inductive
              hypothesis. This means that the $(k -2) + 3 = k+1$ domino will fall. We have
              shown that if the $k^{th}$ domino falls then the $k +1$ domino will fall and Thus
              the statement is true by the principle of strong induction.
          \end{proof}
    \item[4.] Let $P(n)$ be the statement that a postage of n cents can be
          formed using just 4-cent stamps and 7-cent stamps. The
          parts of this exercise outline a strong induction proof that
          $P(n)$ is true for $n \geq  18$.
          \begin{enumerate}[a.]
              \item Show statements $P(18)$, $P(19)$, $P(20)$, and $P(21)$ are true, completing the basis step of the proof.
              \item What is the inductive hypothesis of the proof?
              \item What do you need to prove in the inductive step?
              \item Complete the inductive step for $k \geq 21$.
              \item Explain why these steps show that this statement is true whenever $n \geq 18$.
          \end{enumerate}
          \answer
          \begin{enumerate}[a.]
              \item $P(18)$ is true because you can use two 7-cent stamps and a 4-cent stamp\\
                    $P(19)$ is true with three 4-cent stamps and a 7-cent stamp.\\
                    $P(20)$ is true with five 4-cent stamps. \\
                    $P(21)$ is true with three 7-cent stamps.
              \item Inductive Hypothesis: Assume that $P(j)$ is true with $18 \leq j \leq k$ with $k \geq 21$
              \item Need to prove that if $P(k)$ is true then $P(k+1)$ is also true with $k \geq 18$
              \item If $k \geq 21$ we know that $P(k-3)$ is true since $k-3 \geq 18$ by the inductive hypothesis
                    This means that $P(k+1)$ is true because we can add a 4-cent coin to the combination from $P(k-3)$.
              \item  Completed basis step and inductive step so it is true for all integers greater than 18.
          \end{enumerate}

\end{itemize}
\subsection{5.3}
\begin{itemize}
    \item[2.]  Find $f(1)$, $f(2)$, $f(3)$, $f(4)$, and $f(5)$ if f (n) is defined
          recursively by $f(0) = 3$ and for $n = 0, 1, 2, \ldots$
          \begin{enumerate}[a.]
              \item $f (n + 1) = -2f (n)$.
              \item $f (n + 1) = 3f (n) + 7$.
              \item $f (n + 1) = f (n)^2 - 2f (n) - 2$.
              \item $f (n + 1) = 3^{f(n)/3}$.
          \end{enumerate}
          \answer
          \begin{enumerate}[a.]
              \item $f(1)=-6, \ f(2)=12, \ f(3) -24, \ f(4) = 48, \ f(5) = -96$
              \item $f(1)=16, \ f(2)=55, \ f(3)= 172 , \ f(4) = 523, \ f(5) = 1576$
              \item $f(1)=1,  \ f(2)=-3, \ f(3)= 13, \ f(4) = 141, \ f(5) = 19597$
              \item $f(1)=3,  \ f(2)=3, \ f(3) =3, \ f(4) = 3, \ f(5) = 3$
          \end{enumerate}
    \item[4.]  Find $f(2),\ f (3),\ f (4),\ \text{and } f (5)$ if $f$ is defined recursively by $f (0) = f (1) = 1$ and for $n = 1,\ 2,\ \ldots$
          \begin{enumerate}[a.]
              \item $f (n + 1) = f (n) - f (n - 1)$.
              \item $f (n + 1) = f (n)f (n - 1)$.
              \item $f (n + 1) = f (n)^2 + f (n - 1)^3 $.
              \item $f (n + 1) = f (n)/f (n - 1)$.
          \end{enumerate}
          \answer
          \begin{enumerate}
              \item $f(2)=0, \ f(3) -1, \ f(4) = -1, \ f(5) = 0$
              \item $f(2)=1, \ f(3) 1, \ f(4) = 1, \ f(5) = 1$
              \item $f(2)=2, \ f(3) 5, \ f(4) = 33, \ f(5) = 1214$
              \item $f(2)=1, \ f(3) 1, \ f(4) = 1, \ f(5) = 1$
          \end{enumerate}
    \item[8.]  Give a recursive definition of the sequence $\{a_n \},\ n = 1,\ 2,\ 3,\ \ldots\ $if
          \begin{enumerate}[a.]
              \item$a_n = 4n - 2.$
              \item$a_n = 1 + (-1)^n $.
              \item$a_n = n(n + 1)$.
              \item$a_n = n^2 $.
          \end{enumerate}
          \answer
          \begin{enumerate}[a.]
              \item $f(1) =2 \\ f(n+1) = f(n) +4$
              \item $f(1)=1\\f(n+1) = f(n) + (-1)^{f(n)}$
              \item $f(1) =2,\ f(2) = 6 \\ f(n+1) = 2f(n) -f(n-1) +2$
              \item $f(1) =1,\ f(2) = 4 \\ f(n+1) = 2f(n) -f(n-1) +2$
          \end{enumerate}


\end{itemize}

\subsection{5.4}
\begin{itemize}
    \item[2.] Trace Algorithm 1 when it is given $n = 6$ as input. That
is, show all steps used by Algorithm 1 to find $6!$, as is
done in Example 1 to find $4!$. 

\answer 

$6! = 6 \cdot 5!, \ 5! = 5 \cdot 4!, \ 4! = 4 \cdot 3!, \ 3! = 3 \cdot 2!, \ 2! = 2 \cdot 1!, \ 1! = 1 \cdot 0!$\\
$0! = 1$ So $1! = 1 \cdot 1 = 1, \ 2! = 2 \cdot 1! = 2, \ 3! = 3 \cdot 2! = 6, \ 4! = 4 \cdot 3! = 24, \ 5! = 5 \cdot 4! = 120 \ 6! = 6 \cdot 5! = 720$

\item[8.]  Give a recursive algorithm for finding the sum of the
first $n$ positive integers.

\answer

\textbf{procedure:} \textit{sum\_to\_n}($n$: nonnegative integer)\\
\textbf{if} n = 0 \textbf{then return} 0\\
\textbf{else return} $n+$ \textit{sum\_to\_n}($n-1$)



\end{itemize}