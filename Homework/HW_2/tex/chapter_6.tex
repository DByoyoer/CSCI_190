\section{Chapter 6}
\subsection{6.1}
\begin{itemize}
    \item[2.]  An office building contains 27 floors and has 37 offices
          on each floor. How many offices are in the building? \\
          \answer \\
          $27 \cdot 37 = 999$

    \item[8.]  How many different three-letter initials with none of the
          letters repeated can people have? \\
          \answer \\
          $26 \cdot 25 \cdot 24 = 15600$

    \item[30.]  How many license plates can be made using either three
          uppercase English letters followed by three digits or four
          uppercase English letters followed by two digits? \\
          \answer\\
          $26^3 \cdot 10^3 + 26^4 \cdot 10^2 = 63273600$

    \item[40.] How many subsets of a set with 100 elements have more
          than one element?\\
          \answer \\
          $2^{100} -101$

    \item[44.]  How many ways are there to seat four of a group of ten
          people around a circular table where two seatings are con-
          sidered the same when everyone has the same immediate
          left and immediate right neighbor? \\
          \answer \\
          $\frac{10 \cdot 9 \cdot 8 \cdot 7}{4} = 1260$

\end{itemize}

\subsection{6.2}
\begin{itemize}
    \item[2.]  Show that if there are 30 students in a class, then at least
          two have last names that begin with the same letter. \\
          \answer \\
          Since there are on 26 letters in the English alphabet, then if there are more
          than 26 students then at least 2 will have the same first letter in there last name.

    \item[4.] A bowl contains 10 red balls and 10 blue balls. A woman
          selects balls at random without looking at them.
          \begin{enumerate}[a.]
              \item How many balls must she select to be sure of having at least three balls of the same color?
              \item How many balls must she select to be sure of having at least three blue balls?
          \end{enumerate}
          \answer
          \begin{enumerate}[a.]
              \item 13 \\
              {\color{red} This is wrong, should be 5 balls, $\lceil 5/2 \rceil = 3$ balls that have the same color.}
              \item 13
          \end{enumerate}

    \item[8.]  Show that if $f$ is a function from $S$ to $T$ , where $S$ and $T$
          are finite sets with $|S| > |T|$, then there are elements $s_1$
          and $s_2$ in $S$ such that $f(s_1 ) = f (s_2)$, or in other words, $f$
          is not one-to-one. \\
          \answer \\
          There are more elements in the domain than the codomain so by the pidgeon hole
          principle there must be 2 or more elements from the domain that map to the
          same element in the codomain thus the functin can't be one-to-one.

    \item[18.]  Suppose that there are nine students in a discrete mathe-
          matics class at a small college.
          \begin{enumerate}[a.]
              \item Show that the class must have at least five male students or at least five female students.
              \item Show that the class must have at least three male students or at least seven female students.
          \end{enumerate}
          \answer
          \begin{enumerate}[a.]
              \item If there are 4 of one there must be 5 of the other.
              \item If there are only 2 male students then there must be $9 - 2 = 7 \geq 7$ female students
                    and if there are only 6 female students then there must be $9 - 6 = 3 \geq 3$ male students.
          \end{enumerate}

\end{itemize}


\subsection{6.3}
\begin{itemize}
    \item[4.]Let $S = \{1, 2, 3, 4, 5\}$.
          \begin{enumerate}[a.]
              \item List all the 3-permutations of S.
              \item List all the 3-combinations of S.
          \end{enumerate}
          \answer
          \begin{enumerate}[a.]
              \item 123, 124, 125, 132, 134, 135, 142, 143, 145, 152, 153, 154,
                    213, 214, 215, 231, 234, 235, 241, 243, 245, 251, 253, 254, 312, 314, 315,
                    321, 324, 325, 341, 342, 345, 351, 352, 354, 412, 413, 415, 421, 423, 425,
                    431, 432, 435, 451, 452, 453, 512, 513, 514, 521, 523, 524, 531, 532, 534,
                    541, 542, 543
              \item 123, 124, 125, 134, 135, 145, 234, 235, 245, 345
          \end{enumerate}

    \item[6.]  Find the value of each of these quantities.
          \begin{enumerate}
              \item C(5, 1)
              \item C(5, 3)
              \item C(8, 4)
              \item C(8, 8)
              \item C(8, 0)
              \item C(12, 6)
          \end{enumerate}
          \answer
          \begin{enumerate}[a.]
              \item 5
              \item 10
              \item 70
              \item 1
              \item 1
              \item 924
          \end{enumerate}

    \item[10.]  There are six different candidates for governor of a state.
          In how many different orders can the names of the can-
          didates be printed on a ballot? \\
          \answer \\
          $6! = 720$ different orders.

    \item[12.] How many bit strings of length 12 contain
\begin{enumerate}[a.]
    \item exactly three 1s?
    \item at most three 1s?
    \item at least three 1s?
    \item an equal number of 0s and 1s?
\end{enumerate}
\answer
\begin{enumerate}[a.]
    \item C(12, 3) = 220
    \item C(12, 0) + C(12, 1) + C(12, 2) + C(12, 3) = 299
    \item $\sum_{n=3}^{12} C(12, n) = 4017$
    \item C(12, 6) = 924
\end{enumerate}
\end{itemize}


\subsection{6.4}
\begin{itemize}
    \item[2.]  Find the expansion of $(x + y)^5$
          \begin{enumerate}[a.]
              \item using combinatorial reasoning, as in Example 1.
              \item using the binomial theorem.
          \end{enumerate}
          \answer
          \begin{enumerate}[a.]
              \item So $(x+y)^5 = (x+y)(x+y)(x+y)(x+y)(x+y)$ gives, $x^5,\ x^4y,\ x^3y^2,\ x^2y^3,\ xy^4,\ \text{and } y^5$ as the terms. 
              To get $x^5$ and $y^5$ the only way is to chose $x$ from all 5 or $y$ 
              from all 5 so they will both have 1 as their coefficient. To get $x^4y$ 
              x must be chosen from 4 of the 5 sums so it is 5 choose 4 which is 5, 
              so $x^4y$ must have a coefficient of 5 and then same for $xy^4$ by the same
              argument for the $y$. Then for $x^3y^2$ and $x^2y^3$ we choose 3 or 2 from
              the 5 sums, which is 10 so those will have coefficients of 10.
              \item Binomial Theorem:
              \begin{equation*}
                  (x+y)^n = \sum_{j=0}^{n} {n \choose j}x^{n-j}y^j 
              \end{equation*}
              So it's ${5 \choose 0} = 1$ for $x^5$, ${5 \choose 1} = 5$ for $x^4y$, ${5 \choose 2} = 10$ for $x^3y^2$, ${5 \choose 3} = 10$ for $x^2y^3$, 
              ${5 \choose 4} = 5$ for $xy^4$ and ${5 \choose 5} =1$ for $y^5$.
          \end{enumerate}
\item[6.]  What is the coefficient of $x^7$ in $(1 + x)^{11}$? \\
\answer \\
${11 \choose 7} = 330$

\item[8.] What is the coefficient of $x^8y^9$ in the expansion of
$(3x + 2y)^{17}$? \\
\answer \\
${17 \choose 8} \cdot 3 \cdot 2 = 145860$

\item[12.]  The row of Pascal’s triangle containing the binomial coefficients ${10 \choose k}, 0 \leq k \leq 10$ \\
1 10 45 120 210 252 210 120 45 10 1 \\
Use Pascal’s identity to produce the row immediately fol-
lowing this row in Pascal’s triangle. \\
\answer \\
1 11 55 165 330 362 362 330 165 55 11 1

\end{itemize}