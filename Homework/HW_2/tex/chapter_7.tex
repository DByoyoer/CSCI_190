\section{Chapter 7}
\subsection{7.1}
%TODO: 7.1 – 2, 8, 30, 40
\begin{itemize}
    \item[2.]  What is the probability that a fair die comes up six when
          it is rolled? \\
          \answer \\
          $1/6$

    \item[8.]  What is the probability that a five-card poker hand contains the ace of hearts? \\
          \answer \\
          In any group of 5 cards each card has a 1/52 chance of being the ace of hearts, so for
          a group of 5 there is $5 \cdot 1/52 = 5/52$ chance of there being an ace of hearts in
          that group.

    \item[30.] What is the probability that a player of a lottery wins
          the prize offered for correctly choosing five (but not six)
          numbers out of six integers chosen at random from the
          integers between 1 and 40, inclusive? \\
          \answer \\
          We know that there is $40 \choose 6$ choices for the 6 numbers. Then we have
          to choose 5 correct numbers out of 6. Then finally there is 34 numbers that we
          can choose that are not the correct one. So the answer is ${6 \choose 5} \cdot 34 / {40 \choose 6} = 5.31 \cdot 10^{-5}$

    \item[40.] Suppose that instead of three doors, there are four doors
          in the Monty Hall puzzle. What is the probability that you
          win by not changing once the host, who knows what is
          behind each door, opens a losing door and gives you the
          chance to change doors? What is the probability that you
          win by changing the door you select to one of the two
          remaining doors among the three that you did not select? \\
          \answer \\
          Not changing doors : 0.25 \\
          Changing doors : 3/8 = 0.375

\end{itemize}

\subsection{7.2}
\begin{itemize}
    %! ALERT: CHECK THIS TONIGHT BEFORE TURNING IN
    \item[2.]  Find the probability of each outcome when a loaded die
          is rolled, if a 3 is twice as likely to appear as each of the
          other five numbers on the die. \\
          \answer \\
          P(3) = 2/5 and the rest is 1/15 each. \\
          {\color{red} This is wrong should be P(1) = P(2) = P(4) = P(5) = P(6) = x, then P(3) = 2x then 7x = 1, x = 1/7, P(3) = 2/7)}

    \item[6.]  What is the probability of these events when we randomly
          select a permutation of {1, 2, 3}?
          \begin{enumerate}[a.]
              \item 1 precedes 3.
              \item 3 precedes 1.
              \item 3 precedes 1 and 3 precedes 2.
          \end{enumerate}
          \answer \\
          Permutations: 123, 132, 213, 231, 312, 321
          \begin{enumerate}[a.]
              \item 1/2
              \item 1/2
              \item 1/3
          \end{enumerate}
    \item[12.] Suppose that $E$ and $F$ are events such that $p(E) =
              0.8 \text{ and } p(F) = 0.6$. Show that $p(E \cup F ) \geq 0.8$ and
          $p(E \cap F) \geq 0.4.$ \\
          \answer \\
          $p(E \cup F) = p(E) + p(F) - p(E \cap F) = 1.4 - p(E \cap F)$ \\
          But since the probability of an event can't be greater than 1 we know that
          $p(E \cap F) \geq  0.4$ \\
          Now to find the min of $p(E \cup F)$ we need to find the max of $p(E \cap F)$
          it can't be higher than the probability of either of the events probability so
          it can't be greater than 0.6. So with the previous formula $p(E \cup F) = 1.4 -0.6 = 0.8$
          as the min. So $p(E \cup F) \geq 0.8$.

    \item[24.] What is the conditional probability that exactly four heads
          appear when a fair coin is flipped five times, given that
          the first flip came up tails? \\
          \answer \\
          $p(E | F) = p(E \cap F) / p(F)$\\
          Let $F$ be first flip is tails then $p(F) = 1/2$ \\
          $p(E \cap F) = 1/2^5$ since there is only 1 sequence of flips that is 1 tails
          out of $2^5$ combinations. So $p(E|F) = (1/32)/(1/2)= 1/16$

    \item[26.]  Let $E$ be the event that a randomly generated bit string
          of length three contains an odd number of 1s, and let $F$
          be the event that the string starts with 1. Are $E$ and $F$
          independent? \\
          \answer \\
          $p(E) = \left({3 \choose 1} + {3 \choose 3}\right) /2^3 = 1/2$ \\
          $p(F) = 2^2/2^3= 1/2$ \\
:          $p(E \cap F) = 2/8 = 1/4 = p(E) \cdot p(F)$ \\
          Thus $E$ and $F$ are independent.
\end{itemize}
